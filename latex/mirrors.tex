\documentclass{beamer}
\usetheme{Warsaw}
\setbeamertemplate{headline}{}
\usepackage[slantfont,boldfont]{xeCJK}
\usepackage{xunicode}
\setCJKmainfont{文泉驿正黑}
\usepackage{xcolor}
\usepackage{verbatim}
\usepackage{url}
\usepackage{array}

\newcommand\graph[1]{{\includegraphics[width=4.5in]{../plot/ps/#1.ps}}}

\begin{document}

\title{Visualizing Mirrors}
\subtitle{mirrors.ustc.edu.cn服务器日志分析}
\author{李博杰 bojieli@gmail.com}
\institute{\copyright USTC LUG}
\date{\today}
\frame{\titlepage}

\AtBeginSection[] {
  \begin{frame}<beamer>{Outline}
  \tableofcontents[currentsection]
  \end{frame}
}

\section{Requests \& Traffic}

\begin{frame}{Notes}
\begin{itemize}
  \item The data is access log of mirrors.ustc.edu.cn in 51 days. See `Technical Details' section for more info about dataset.
  \item Some graphs are in log-scale for clarity. Please note whether $x$ axis, $y$ axis or both are in log-scale. The graph title sometimes lies.
  \item Because there may be many points in a graph, sampling is made to reduce file size (they are vector graphics), hence there may be some `straight lines'. I have checked the data to make sure the graphs illustrate real trends.
  \item Title length is limited, so the title itself may not explain well, please keep an eye on the axis and keys of the graph.
  \item Graphs are shown in the hope of conveying information without words. Any questions or suggestions, please email me.
\end{itemize}
\end{frame}

\subsection{By Time}

\begin{frame}{Requests \& Traffic in a day}
\graph{req-daymin}
\end{frame}

\begin{frame}{Requests \& Traffic in a week}
\graph{req-weekday}
\end{frame}

\begin{frame}{Requests \& Traffic across 50 days}
\graph{req-days}
\end{frame}

\begin{frame}{Statistics}
\begin{tabular}[t]{|l|r|r|}
\hline
 & Requests & Traffic \\
\hline
Total & 328976877 & 36892 GB \\
Avg. per Day & 6450527 & 723.4 GB \\
Max. per Day & 8632963 & 1049.5 GB \\
Min. per Day & 4868022 & 421.5 GB \\
Avg. per Hour & 268771 & 30.14 GB \\
Max. per Hour & 561925 & 79.75 GB \\
Min. per Hour & 99506 & 2.97 GB \\
Avg. per Minute & 4480 & 514.4 MB \\
Max. per Minute & 14714 & N/A \\
Min. per Minute & 441 & N/A \\
Avg. per Second & 74.66 & 8779 KB \\
Max. per Second & 2117 & N/A \\
Min. per Second & 1 & N/A \\
\hline
\end{tabular}
\begin{itemize}
  \item \tiny{Because the time recorded is only completion time of the request, and large requests can span hours, so Max./Min. per minute/second is not applicable.}
\end{itemize}
\end{frame}

\begin{frame}{Cumulative Requests per Hour}
\graph{req-acc-hour}
\end{frame}

\begin{frame}{Cumulative Requests per Minute}
\graph{req-acc-min}
\end{frame}

\begin{frame}{Cumulative Requests per Second}
\graph{req-acc-sec}
\end{frame}


\subsection{By IP}

\begin{frame}{Cumulative Traffic over IPs: 20\%-80\% law}
\graph{traffic-acc-ip}
\end{frame}

\begin{frame}{Cumulative Requests over IPs: 20\%-80\% law}
\graph{req-acc-ip}
\end{frame}

\begin{frame}{IPv4 vs. IPv6}
\begin{tabular}[t]{|l|r|r|}
\hline
 & Requests & Traffic \\
\hline
IPv4 & 318575688 (96.84\%) & 34180 GB (92.65\%) \\
IPv6 & 10401189 (3.15\%) & 2712 GB (7.35\%) \\
\hline
\end{tabular}
\begin{itemize}
  \item It can be seen that IPv6 still have a long way to go\ldots
\end{itemize}
\end{frame}

\begin{frame}{Requests \& Traffic TOP 40: xxx.0.0.0/24}
\graph{req-ipv4-0}
\end{frame}

\begin{frame}{Traffic TOP 40: IPv4 addrs}
\graph{req-ipv4-top40}
\end{frame}

\begin{frame}{Request count TOP 40: IPv4 addrs}
\graph{req-ipv4-top40-req}
\end{frame}

\begin{frame}{USTC Mirrors Usage (IPv4 only)}
\begin{tabular}[t]{|>{\tiny}l|>{\small}r|>{\small}r|>{\tiny}r|}
\hline
IP range & Requests & Traffic & Note \\
\hline
202.38.64.0-202.38.95.255 & 994661 (0.30\%) & 149.3 GB (0.40\%) & CERNET \\
210.45.64.0-210.45.79.255 & 191141 (0.06\%) & 68.31 GB (0.19\%) & CERNET \\
210.45.112.0-210.45.127.255 & 243976 (0.07\%) & 37.59 GB (0.10\%) & CERNET \\
211.86.144.0-211.86.159.255 & 81035 (0.02\%) & 24.96 GB (0.07\%) & CERNET \\
222.195.64.0-222.195.95.255 & 319435 (0.10\%) & 86.88 GB (0.24\%) & CERNET \\
114.214.160.0-114.214.255.255 & 0 & 0 & CERNET \\
210.72.22.0-210.72.22.255 & 3622 (0.00\%) & 11.86 MB (0.00\%) & TechNet (?) \\
218.22.21.0-218.22.21.31 & 1 (0.00\%) & 0.01 MB (0.00\%) & China Telecom \\
218.104.71.160-218.104.71.175 & 0 & 0 & China Unicom \\
202.141.160.0-202.141.175.255 & 123455 (0.04\%) & 12.60 GB (0.03\%) & China Telecom \\
202.141.176.0-202.141.191.255 & 187 (0.00\%) & 120.4 MB (0.00\%) & China Mobile \\
Total & 1957513 (0.60\%) & 379.77 GB (1.03\%) & USTC IPv4 \\
\hline
\end{tabular}
\begin{itemize}
  \item \tiny{Data source of USTC IP range: http://lib.ustc.edu.cn/ustcip.html}
\end{itemize}
\end{frame}

\subsection{By Other Measures}

\begin{frame}{Requests \& Traffic of distributions}
\graph{req-dist}
\end{frame}

\begin{frame}{Requests \& Traffic of distributions}
\graph{req-dist-req}
\end{frame}


\begin{frame}{Requests \& Traffic among HTTP Status Codes}
\graph{req-status-code}
\end{frame}


\begin{frame}{Requests \& Traffic among User Agents}
\graph{req-user-agent}
\end{frame}


\begin{frame}{Traffic per Request order by Length}
\graph{req-traffic}
\end{frame}

\begin{frame}{Cumulative Traffic sorted by Request Length}
\graph{req-traffic-acc}
\end{frame}

\begin{frame}{Request Num sorted by Request Length}
\graph{req-traffic-all}
\end{frame}


\section{Files}

\subsection{Files Characteristics}

\begin{frame}{File Number at different FileSizes (log scale)}
\graph{file-size-all}
\end{frame}

\begin{frame}{Total Size at different Filesizes (log scale)}
\graph{file-size-total}
\end{frame}

\begin{frame}{Cumulative Filesize per File Num (normal scale)}
\graph{file-num-size}
\end{frame}

\begin{frame}{Cumulative Filesize per File Num (log scale)}
\graph{file-num-size-log}
\end{frame}

\begin{frame}{Cumulative Filesize per FileSize (normal scale)}
\graph{file-size-acc-nologscale}
\end{frame}

\begin{frame}{Cumulative Filesize per FileSize (log scale)}
\graph{file-size-acc}
\end{frame}

\begin{frame}{File Extensions TOP 40 (order by Total Size)}
\graph{file-ext-top40}
\end{frame}

\begin{frame}{File Extensions TOP 40 (order by File Num)}
\graph{file-ext-top40-count}
\end{frame}

\begin{frame}{How many files share a same filename (extension excluded)}
\graph{file-same-filename}
\end{frame}

\begin{frame}{How many files share a same filename and extension}
\graph{file-same-filename-ext}
\end{frame}

\begin{frame}{Num \& Size of Files in each Distribution (sorted by Size)}
\graph{file-dist}
\end{frame}

\begin{frame}{Num \& Size of Files in each Distribution (sorted by Num)}
\graph{file-dist-num}
\end{frame}

\subsection{How Files Are Requested}

\begin{frame}{Cumulative Requests over FileSize order by Size}
\graph{file-req-all}
\end{frame}

\begin{frame}{Cumulative Traffic over non-cumu. FileSize}
\graph{file-traffic-all}
\end{frame}

\begin{frame}{Cumulative Traffic over non-cumu. Filesize (log-scale)}
\graph{file-traffic-all-log}
\end{frame}

\begin{frame}{Cache 1: Cumu. Traffic over FileSize order by Size}
\graph{file-traffic-size-acc}
\end{frame}

\begin{frame}{Cache 2: Cumu. Traffic over FileSize order by Size DESC}
\graph{file-traffic-size-desc-acc}
\end{frame}

\begin{frame}{Cache 3: Cumu. Traffic over FileSize order by Req Num}
\graph{file-traffic-req-acc}
\end{frame}

\begin{frame}{Cache 4: Cumu. Traffic over FileSize order by Traffic}
\graph{file-traffic-traffic-acc}
\end{frame}

\begin{frame}{Cache 5: Cumu. Traffic order by Traffic/FileSize}
\graph{file-traffic-ratio-acc}
\end{frame}

\begin{frame}{Comparison of the Previous five `Caching Policies'}
\graph{file-traffic-five-acc}
\end{frame}

\begin{frame}{Comparison of `Caching Policies' (log-scale)}
\graph{file-traffic-five-acc-log}
\end{frame}

\begin{frame}{Comparison of `Caching Policies'}
\begin{itemize}
  \item Among these static caching policies, caching files with most traffic or requests is acceptable.
  \item Caching files that carried most traffic in history has a good performance. A 10GB cache of 85 files can cover 40\% of the total traffic. If cache size continue to increase, the cache efficiency will deteriorate, since x axis of the graph is in log-scale.
  \item Caching files with largest traffic/filesize ratio shows best performance (by definition). A 10GB cache of 7162 files can cover 58\% of the total traffic.
\end{itemize}
\end{frame}

\begin{frame}{Details of Most Traffic Caching (log-scale)}
\graph{file-traffic-traffic-filenum}
\end{frame}

\begin{frame}{Details of Traffic/FileSize Caching (log-scale)}
\graph{file-traffic-ratio-filenum}
\end{frame}

\begin{frame}{An Alternate Metric: Request Hits}
\graph{file-req-four-acc}
\end{frame}

\begin{frame}{An Alternate Metric: Request Hits}
\graph{file-req-four-acc-log}
\end{frame}

\begin{frame}{Num \& Size of Never-accessed Files (\% of total)}
\graph{file-dist-never-req}
\end{frame}

\begin{frame}{Num \& Size of Never-accessed Files (\% of total)}
\graph{file-dist-never-req-traffic}
\end{frame}

\begin{frame}{Num \& Size of Never-accessed Files (\% of distribution)}
\graph{file-dist-never-req-percent}
\end{frame}

\begin{frame}{Num \& Size of Never-accessed Files (\% of distribution)}
\graph{file-dist-never-req-percent-traffic}
\end{frame}

\begin{frame}{Num \& Size of Ever-accessed Files (\% of distribution)}
\graph{file-dist-ever-req-percent}
\end{frame}

\begin{frame}{Num \& Size of Ever-accessed Files (\% of distribution)}
\graph{file-dist-ever-req-percent-traffic}
\end{frame}

\section{Sessions}

\begin{frame}{Discovering Sessions}
\begin{definition}
Two requests are within a same \emph{Interval Session} iff:
\begin{itemize}
  \item Have same IP address
  \item The time difference does not exceed some limit
\end{itemize}
\end{definition}

\begin{definition}
A \emph{Gap Session} is a longest sequence of requests where:
\begin{itemize}
  \item All requests are from the same IP address
  \item Time difference of every two adjacent requests do not exceed some limit
\end{itemize}
\end{definition}
\end{frame}

\begin{frame}{Discovering Sessions (continued)}
\begin{itemize}
  \item Since Mirrors is a resource-downloading site, many sessions download many files for hours. The time limit of Interval Session is 60 minutes.
  \item The time limit of Gap Session is 30 minutes. For long downloads that extend over 30 minutes, the session will be broken into two.
  \item Both algorithms suffer from false positives and true negatives.
  \item Some IPs access so frequently that the gap session never ends (see the next slide), making the Gap Session data highly biased.
\end{itemize}
\end{frame}

\begin{frame}{Average Session Duration over IP (log-scale)}
\graph{session-avg-duration-ip}
\end{frame}

\begin{frame}{Average Session Duration over IP (normal-scale)}
\graph{session-avg-duration-ip-nolog}
\end{frame}

\begin{frame}{Gap Session Statistics in a day}
\graph{session-daymin-gap}
\end{frame}

\begin{frame}{Explanation}
\begin{itemize}
  \item How strange the three graphs look! Durations are always integers (seconds), so there are straight lines in duration graphs.
  \item Many sessions are actually `single' requests, so their duration is zero, and the amount of them can be seen in normal-scale graph.
  \item About 40 sessions extend throughout the whole 51 days, and more sessions extend less days. The amount of them is the length of red horizontal line in log-scale graph.
  \item Because the log starts at May 22 06:25:37, requests in these long-live Gap Sessions accumulate to a horrible peak at 06:27 in Gap Session statistics (much higher than shown, since the original data is noizy, and the graph is Bezier-smoothed).
\end{itemize}
\end{frame}

\begin{frame}{Cumulative Over Long-live Gap Sessions}
\graph{session-longlive}
\end{frame}

\begin{frame}{Ratio to Average for Long-live Gap Sessions (log-scale)}
\graph{session-longlive-avg}
\end{frame}

\begin{frame}{Statistics of Interval Sessions}
\begin{tabular}[t]{|l|r|r|r|r|r|}
\hline
 & Total & Average & Max & Min & Std. Dev \\
\hline
Requests & 328976877 & 7.5020 & 186043 & 1 & 173.1824 \\
Traffic & 36277 GB & 867.46 KB & Overflow & 0 & 15.227 MB \\
Duration & $4.026 * 10^{10}$ & 918.13 s & 3599 s & -6 s & 1502.9 s \\
\hline
\end{tabular}
\begin{itemize}
  \item 43852053 sessions in total.
  \item Sessions with negative duration is because log items can accidentally be not in time non-decreasing order. I'm not going to fix it, since these 230 wrong sessions have little influence.
\end{itemize}
\end{frame}

\begin{frame}{Statistics of Gap Sessions}
\begin{tabular}[t]{|l|r|r|r|r|r|}
\hline
 & Total & Average & Max & Min & Std. Dev \\
\hline
Requests & 328976877 & 6.6200 & 8432330 & 1 & 1255.9488 \\
Traffic & 35291 GB & 744.67 KB & Overflow & 0 & 15.184 MB \\
Duration & $7.666 * 10^{9}$ & 154.27 s & 4406403 & -27 s & 7060.6 s \\
\hline
\end{tabular}
\begin{itemize}
  \item 49694582 sessions in total.
  \item For comparison with Interval Sessions.
  \item The following statistics are based on Interval Session if not noted specially.
\end{itemize}
\end{frame}

\begin{frame}{Cumulatives over Sessions order by Traffic DESC}
\graph{session-req-traffic}
\end{frame}

\begin{frame}{Cumulatives over Sessions order by Traffic DESC}
\graph{session-req-traffic-log}
\end{frame}

\begin{frame}{Cumulatives over Sessions order by Requests DESC}
\graph{session-req-req-log}
\end{frame}

\begin{frame}{Cumulatives over Sessions order by Duration DESC}
\graph{session-req-duration-log}
\end{frame}

\begin{frame}{Cumulative Session Duration over Unique IPs}
\graph{session-int-ip}
\end{frame}

\begin{frame}{Sessions in a day}
\graph{session-daymin}
\end{frame}

\begin{frame}{Sessions across days}
\graph{session-days}
\end{frame}

\begin{frame}{Sessions in Distributions order by Traffic}
\graph{session-dist-traffic}
\end{frame}

\begin{frame}{Sessions in Distributions order by Requests}
\graph{session-dist-req}
\end{frame}

\begin{frame}{Sessions in Distributions order by Session Num}
\graph{session-dist-num}
\end{frame}

\begin{frame}{Sessions in Distributions order by Duration}
\graph{session-dist-duration}
\end{frame}

\begin{frame}{Sessions and User Agents}
\graph{session-user-agent}
\end{frame}

\section{Distributions Insight}

\subsection{CentOS}

\begin{frame}{CentOS Basic Stat}
\begin{tabular}[t]{|l|r|r|r|}
\hline
 & Value & \% of total & Rank \\
\hline
Requests& 100986986 & 30.7\% & 1 \\
Traffic & 5252.5 GB & 14.2\% & 4 \\
Files 	& 70043 & 0.7\% & 19 \\
FileSize& 172.8 GB & 1.3\% & 12 \\
Sessions& 19452260 & 44.4\% & 1 \\
\hline
\end{tabular}
\end{frame}

\begin{frame}{CentOS Basic Stat - Average}
\begin{tabular}[t]{|l|r|r|r|r|}
\hline
 & Average & Ratio & Std.Dev & Ratio \\
\hline
Request Length	& 55846 & 0.464 & 882920 & 0.289 \\
File Size	& 2665431 & 1.953 & 54320911 & 2.294 \\
File Requests	& 1402.46 & 76.994 & 92327 & 11.029 \\
File Traffic	& 74454404 & 32.12 & 1231957014 & 1.784 \\
Session Duration& 1051.19 & 1.145 & 1586.69 & 1.056 \\
Session Requests& 5.796 & 0.773 & 137.5 & 0.794 \\
Session Traffic & 336791 & 0.379 & 8931320 & 0.559 \\
\hline
\end{tabular}
\begin{itemize}
  \item Std.Dev stands for Standard Deviation.
  \item `Ratio' in the third col stands for Ratio of Distribution Average to Global Average; `Ratio' in the fifth col stands for Ratio of DIst Std.Dev to Global.
  \item Size \& Traffic are in bytes, Duration is in seconds.
\end{itemize}
\end{frame}

\begin{frame}{Requests \& Traffic among CentOS Versions}
\graph{dist-subdir-centos}
\end{frame}

\begin{frame}{Requests \& Traffic among CentOS 2-level Subdirs}
\graph{dist-subsubdir-centos}
\end{frame}

\subsection{Fedora}

\begin{frame}{Fedora Basic Stat}
\begin{tabular}[t]{|l|r|r|r|}
\hline
 & Value & \% of total & Rank \\
\hline
Requests& 36329528 & 11.0\% & 4 \\
Traffic	& 7509.2 GB & 20.3\% & 2 \\
Files 	& 924620 & 8.8\% & 2 \\
FileSize& 1207.8 GB & 9.4\% & 2 \\
Sessions& 3596727 & 8.2\% & 2 \\
\hline
\end{tabular}
\end{frame}

\begin{frame}{Fedora Basic Stat - Average}
\begin{tabular}[t]{|l|r|r|r|r|}
\hline
 & Average & Ratio & Std.Dev & Ratio \\
\hline
Request Length	& 221945 & 1.843 & 2277229 & 0.744 \\
File Size	& 1407150 & 1.031 & 21376814 & 0.903 \\
File Requests	& 28.3134 & 1.554 & 6563 & 0.784 \\
File Traffic	& 3601714 & 1.554 & 190851795 & 0.276 \\
Session Duration& 1226 & 1.336 & 1598 & 1.063 \\
Session Requests& 10.8047 & 1.440 & 251.43 & 1.452 \\
Session Traffic & 2238126 & 2.520 & 21424912 & 1.342 \\
\hline
\end{tabular}
\end{frame}

\begin{frame}{Requests \& Traffic among Fedora Subdirs}
\graph{dist-subdir-fedora}
\end{frame}

\begin{frame}{Requests \& Traffic among Fedora 2-level Subdirs}
\graph{dist-subsubdir-fedora}
\end{frame}

\subsection{Ubuntu}

\begin{frame}{Ubuntu Basic Stat}
\begin{tabular}[t]{|l|r|r|r|}
\hline
 & Value & \% of total & Rank \\
\hline
Requests& 19193451 & 5.8\% & 5 \\
Traffic & 5653.1 GB & 15.3\% & 3 \\
Files 	& 608361 & 5.8\% & 5 \\
FileSize& 584.6 GB & 4.6\% & 6 \\
Sessions& 160800 & 0.4\% & 4 \\
\hline
\end{tabular}
\end{frame}

\begin{frame}{Ubuntu Basic Stat - Average}
\begin{tabular}[t]{|l|r|r|r|r|}
\hline
 & Average & Ratio & Std.Dev & Ratio \\
\hline
Request Length	& 316251 & 2.626 & 2950601 & 0.964 \\
File Size	& 1100038 & 0.806 & 8926254 & 0.377 \\
File Requests	& 22.2714 & 1.222 & 860.6907 & 0.103 \\
File Traffic	& 8152155 & 3.517 & 551019922 & 0.798 \\
Session Duration& 636.37 & 0.693 & 1053.94 & 0.701 \\
Session Requests& 110.93 & 14.788 & 294.68 & 1.702 \\
Session Traffic & 34260641 & 38.570 & 101247464 & 6.341 \\
\hline
\end{tabular}
\end{frame}

\begin{frame}{Requests \& Traffic among Ubuntu Subdirs}
\graph{dist-subdir-ubuntu}
\end{frame}

\begin{frame}{Requests \& Traffic among Ubuntu 2-level Subdirs}
\graph{dist-subsubdir-ubuntu}
\end{frame}

\subsection{Eclipse}

\begin{frame}{Eclipse Basic Stat}
\begin{tabular}[t]{|l|r|r|r|}
\hline
 & Value & \% of total & Rank \\
\hline
Requests& 51279473 & 15.6\% & 3 \\
Traffic & 11498.6 GB & 31.2\% & 1 \\
Files 	& 263355 & 2.5\% & 8 \\
FileSize& 161.1 GB & 1.3\% & 14 \\
Sessions& 524586 & 1.2\% & 3 \\
\hline
\end{tabular}
\end{frame}

\begin{frame}{Eclipse Basic Stat - Average}
\begin{tabular}[t]{|l|r|r|r|r|}
\hline
 & Average & Ratio & Std.Dev & Ratio \\
\hline
Request Length	& 240769 & 2.000 & 5813562 & 1.901 \\
File Size	& 700002 & 0.513 & 8457323 & 0.357 \\
File Requests	& 113.5340 & 6.233 & 17894.2 & 2.137 \\
File Traffic	& 28156070 & 12.148 & 4197033259 & 6.079 \\
Session Duration& 673.17 & 0.733 & 1059.08 & 0.704 \\
Session Requests& 97.1089 & 12.944 & 782.80 & 4.520 \\
Session Traffic & 22130605 & 24.914 & 62746610 & 3.930 \\
\hline
\end{tabular}
\end{frame}

\begin{frame}{Requests \& Traffic among Eclipse Subdirs}
\graph{dist-subdir-eclipse}
\end{frame}

\begin{frame}{Requests \& Traffic among Eclipse 2-level Subdirs}
\graph{dist-subsubdir-eclipse}
\end{frame}

\section{Technical Details}

\begin{frame}{Data Source}
\begin{itemize}
  \item Nginx access log of mirrors.ustc.edu.cn 
  \begin{itemize}
    \item From 2012-05-22 to 2012-07-12, 51 days
    \item 4041MB compressed, 62383MB decompressed
    \item Thanks Guo JiaHua for providing data
  \end{itemize}
  \item File list of mirrors.ustc.edu.cn crawled by spider
  \begin{itemize}
    \item FTP for CPAN and CRAN
    \item HTTP other directories
    \item Need to detect symlinks to parent dir (e.g. /ubuntu/ubuntu/\ldots)
  \end{itemize}
  \item Scripts are written in bash and PHP.
  \begin{itemize}
    \item All scripts are available at GitHub: https://github.com/bojieli/mirrors-log
  \end{itemize}
\end{itemize}
\end{frame}

\begin{frame}{Saving Data in BRIGHTHOUSE}
\begin{itemize}
  \item Scanning through such amount of data is time-consuming. And the data is not too large to fit in a relational database.
  \item I tried InfoBright and InfiniDB with $6.4*10^6$ rows of artificial data, InfiniDB is faster in queries, while InfoBright takes much less disk space.
  \begin{itemize}
    \item InfoBright's compress rate is no less than gzip: 4316MB table size, compared to 4041MB gzip, not to mention that I have added some additional rows for faster statistics.
    \item I do not have much disk space, so I choose InfoBright (a backend of MySQL).
    \item Most of the queries I used (mostly GROUP BY, WHERE) take less than 2 minutes.
  \end{itemize}
  \item Create two FIFOs foreach log: zcat logfile > php (preprocessing) > mysql-ib LOAD DATA INFILE
\end{itemize}
\end{frame}

\begin{frame}{Preprocessing}
\begin{itemize}
  \item Make queries faster (I'm afraid of full-table scan)
  \item ip => integers: ipv4 ipv4\_0 ipv4\_1 ipv4\_2 ipv4\_3 ipv6\_0 ipv6\_1 ipv6\_2 ipv6\_3
  \item time => integers: time year yearday weekday daymin daytime hour
  \item status (200, 403, 404\ldots)
  \item length (filesize)
  \item url => substrings: url\_0 \ldots url\_9 filename extension
  \item referer (I do not want to analyze it, for Mirrors is not a site of user-interaction)
  \item ua => ua\_0 (Mozilla, Ubuntu\ldots), ua (full)
\end{itemize}
\end{frame}

\begin{frame}{Preprocessing}
\begin{itemize}
  \item PHP is \emph{slow} \ldots only 3MB/s.
  \item At first preg\_match (regular exps) take 85\% time, then I optimized the regexp and it only takes 25\% now.
  \item Xdebug show that stream\_get\_line (fgets) and fputs take about 50\% of total time.
  \item InfoBright's data loading speed is 15MB/s (crawl\_http.log). I don't think PHP's preprocessing work is harder than database's\ldots
  \item Maybe PHP's interpretive nature makes it much slower than C. Anyone give a benchmark for Python etc?
\end{itemize}
\end{frame}

\begin{frame}{Files Table}
\begin{itemize}
  \item Many files on mirrors are never accessed, so we have to make a full list of files on mirrors.
  \item Preprocessed url, filename, extension and filesize are recorded for each file.
  \item When processing logs and files, escape characters and VARCHAR max length should be taken care of, and filename should not be limited to a simple regular expression, since there are always exceptions: UTF-8 strings in malicious request, ",v" files in CVS \ldots
\end{itemize}
\end{frame}

\begin{frame}{GNUPLOT}
\begin{itemize}
  \item GNUPLOT is pretty flexible in plotting, the documentation and online demos are consulted many times.
  \item However GNUPLOT is not flexible in processing data. It is strong type, where integers and strings need to be explicitly converted. And the integer type is limited to $-2^{31}$ ~ $2^{31}-1$.
  \item When the query result needs to be postprocessed, I write a simple sed  s/regexp/replace/ for simple replacement, or awk NR\%n==0 for sampling. When it comes to accumulation or some complex stuff, a PHP script goes between.
\end{itemize}
\end{frame}

\begin{frame}{Discovering Sessions}
\begin{itemize}
  \item Sequentially scan the log. If a request can fill into an existing session, update it; otherwise create a new one and flush the timeout session if exists.
  \item Garbage collection of timeout sessions: if (count(array) > gc\_limit) \{ unset timeout sessions; gc\_limit = count(array) * 1.5; \}
  \begin{itemize}
    \item Inspired by JavaScript GC algorithm in IE7: If recycled memory is less than 15\% of total, then limit is doubled; if recycled memory is more than 85\%, then reset to initial value.
  \end{itemize}
  \item Maintain a query buffer of 10000 rows to reduce query num.
  \item The PHP takes 4 hours, 15 times slower than C (see the following section).
\end{itemize}
\end{frame}

\begin{frame}{Some Bugs Due to DIE Time Functions (PHP)}
\begin{itemize}
  \item PHP: Find some sessions with negative session length. The only explanation is that logs are not in time increasing order. Dive into the log table, only to find that a timestamp matches records in both April 31 and May 1.
  \item In fact, mktime() accepts 5 parameters just like mktime(struct time\_t) in C, while its month is 1 to 12, different from 0 to 11. However strptime() is a simple encapsulation of its C respective.
  \item I dare not to use DATETIME in MySQL, because the arithmetic of such timestamps is tricky, and I would rather implement it in SQL or PHP.
  \item Thanks Godness, no timezone problem this time.
\end{itemize}
\end{frame}

\begin{frame}{Some Bugs Due to DIE Time Functions (C)}
\begin{itemize}
  \item C: malloc() fall into deadlock, so strange, I GDBed an evening and no answer. Change some code and accidentally got Segmentation Error in time().
  \item In fact, time() need a parameter of type time\_t and it is dynamically linked. I used it without any parameter, and time() treats the garbage on stack as its parameter. If it is NULL, nothing happens; if not, the position is considered a struct time\_t and unpredictable stuff happen.
\end{itemize}
\end{frame}

\section{Query Optimization}

\begin{frame}[fragile]{Query Optimization}
\begin{itemize}
  \item {\small
\begin{verbatim}
select count(*), ipv4, count(*) as c, sum(length) as s,
concat(ipv4_0,'.',ipv4_1,'.',ipv4_2,'.',ipv4_3) from log
where ipv4 is not null group by ipv4 order by s limit 40;
\end{verbatim}
}
  \item Query\_time: 798.179622
  \begin{itemize}
    \item 2012-07-30 16:32:12 Cnd(0):   VC:0(t0a0) IS NOT NULL       (0)
    \item 2012-07-30 16:33:02 Aggregating: 318575688 tuples left.
    \item 2012-07-30 16:45:29 Aggregated (1415554 gr). Omitted packrows: 0 + 0 partially, out of 5020 total.
    \item 2012-07-30 16:45:29 Heap Sort initialized for 1415554 rows, 8+61 bytes each.
    \item 2012-07-30 16:45:30 Total data packs actually loaded (approx.): 35139
  \end{itemize}
  \item Infobright is column-based, so cross-column queries are slow. Try to generate IP string from `ipv4' field.
\end{itemize}
\end{frame}

\begin{frame}[fragile]{Query Optimization (continued)}
\begin{itemize}
  \item {\small
\begin{verbatim}
SELECT ipv4, COUNT(*)/$COUNT, SUM(length)/$LENGTH AS c,
CONCAT((ipv4 & 255<<24)>>24, '.', (ipv4 & 255<<16)>>16, '.',
(ipv4 & 255<<8)>>8, '.', (ipv4 & 255)) FROM log WHERE ipv4
IS NOT NULL GROUP BY ipv4 ORDER BY c DESC LIMIT 40;
\end{verbatim}
}
  \item Query\_time: 585.069205
  \item InfoBright packs column data into packages, and pre-computed MAX, MIN, AVG and GROUP data for each package. WHERE clause requires re-computation of these data:
  \begin{itemize}
    \item 2012-07-30 14:33:06 Cnd(0):   VC:0(t0a0) IS NOT NULL        (0)
    \item 2012-07-30 14:33:55 Aggregating: 318575688 tuples left.
    \item 2012-07-30 14:42:47 Generating output.
  \end{itemize}
\end{itemize}
\end{frame}

\begin{frame}[fragile]{Query Optimization (continued)}
\begin{itemize}
  \item {\small
\begin{verbatim}
SELECT ipv4, COUNT(*)/$COUNT, SUM(length)/$LENGTH AS c,
CONCAT((ipv4 & 255<<24)>>24, '.', (ipv4 & 255<<16)>>16, '.',
(ipv4 & 255<<8)>>8, '.', (ipv4 & 255)) FROM log GROUP BY
ipv4 ORDER BY c DESC LIMIT 40;
\end{verbatim}
  }
  \item Query\_time: 328.580979
  \item WHERE clause is removed, but the result is wrong (includes a large NULL which stands for IPv6).
  \begin{itemize}
    \item 2012-07-30 14:45:41 Unoptimized expression near '/'
    \item 2012-07-30 14:45:41 Unoptimized expression near '/'
    \item 2012-07-30 14:45:41 Unoptimized expression near 'concat'
    \item 2012-07-30 14:45:41 Aggregating: 328976877 tuples left.
  \end{itemize}
  \item InfoBright calculates these columns for each tuple, no wonder it is slow. Keep the core subquery small. Moving these calculation outside may help.
\end{itemize}
\end{frame}

\begin{frame}[fragile]{Query Optimization (continued)}
\begin{itemize}
  \item {\small
\begin{verbatim}
SELECT ipv4, c/$COUNT, s/$LENGTH, CONCAT((ipv4 &
255<<24)>>24, '.', (ipv4 & 255<<16)>>16, '.', (ipv4 &
255<<8)>>8, '.', (ipv4 & 255)) FROM (SELECT ipv4,
COUNT(*) AS c, SUM(length) AS s FROM log GROUP BY ipv4
HAVING (ipv4 IS NOT NULL) ORDER BY s DESC LIMIT 40) AS t;
\end{verbatim}
}
  \item Query\_time: 138.297943
  \begin{itemize}
    \item Total data packs actually loaded (approx.): 10040
  \end{itemize}
  \item Use HAVING clause to filter NULL. The internal exec order is: FROM TABLE, JOIN, OUTER JOIN, WHERE, SELECT clause, GROUP BY, HAVING, ORDER BY, LIMIT, projection \& output.
  \item Turn on MySQL's query\_cache: identical queries should not be re-executed if I change some GNUPLOT command and run the script again.
  \item Is there any way to make the query faster? I don't know.
\end{itemize}
\end{frame}

\begin{frame}[fragile]{Query Optimization (continued)}
\begin{itemize}
  \item Another example of moving expressions `outside' inner query:
  \item {\small
\begin{verbatim}
SELECT ipv4_0, c/$COUNT, s/$LENGTH FROM (SELECT
IFNULL(ipv4_0, 'IPv6') AS ipv4_0, COUNT(*) AS c,
SUM(length) AS s FROM log GROUP BY ipv4_0 ORDER BY s
DESC LIMIT 40) AS t;
\end{verbatim}
}
  \item Query\_time: 263.048662
  \item {\small
\begin{verbatim}
SELECT IFNULL(ipv4_0, 'IPv6'), c/$COUNT, s/$LENGTH
FROM (SELECT ipv4_0, COUNT(*) AS c, SUM(length) AS s
FROM log GROUP BY ipv4_0 ORDER BY s DESC LIMIT 40) AS t;
\end{verbatim}
}
  \item Query\_time: 84.589837
  \item I cannot believe my eyes at the first sight of these figures.
\end{itemize}
\end{frame}

\begin{frame}[fragile]{Query Optimization is Not Everything}
\begin{itemize}
  \item {\small
\begin{verbatim}
SELECT url_0, COUNT(*)/$COUNT, SUM(files.filesize)/$LENGTH
AS c FROM files WHERE NOT EXISTS (SELECT * FROM log WHERE
log.filename = files.filename) GROUP BY url_0 ORDER BY c
DESC LIMIT 40;
\end{verbatim}
}
  \item The query ran 12 hours and I have to kill it. Infobright treats NOT EXISTS clause as dependent subquery and might have to execute it for each row.
  \item I don't know how to use cursors and storage procedure in MySQL, and more importantly Infobright does not support dynamic update of tables, so SELECT INTO is impossible. 
  \item I write a C program (using mysqlclient API) to count occurences of each url in log and write it into another table.
  \item Performance:
  \begin{itemize}
    \item 10M rows in files table, 329M rows in log table
    \item 17.4 minutes in total
    \item 440000 rows per second at Probe phase
    \item 460 MB of memory usage (360MB resident)
  \end{itemize}
\end{itemize}
\end{frame}

\begin{frame}{Simulating Hash JOIN}
\begin{itemize}
  \item I found my algorithm actually named Hash JOIN after programming finished. It is a fast JOIN algorithm for RDBMS. The task is to find, for each distinct value of the join attribute, the set of tuples in each relation which have that value. (Wikipedia)
  \item My work is to count the occurence in log table of each url in files table.
  \begin{itemize}
    \item Build step: Traverse files table and add the url field of each row to a hash list.
    \item Probe step: Traverse log table and add the counter of the corresponding hash slot.
    \item Output step: Traverse files table again and output the corresponding counters. The output is FIFOed to LOAD DATA INFILE.
  \end{itemize}
  \item Easy to shard horizontally. Steps are similar to Map, Shuffle and Reduce.
\end{itemize}
\end{frame}

\begin{frame}{Simulating Hash JOIN (continued)}
\begin{itemize}
  \item Since there might be many files (currently only about 10M), storing the hash list entirely in memory is my top concern.
  \item Use one hash for position, two additional hashes for checking, and do not store the original string. $10^{-22}$ probability of undetected hash collision, but it is low enough for an analytic program.
  \item Only requires 12 bytes for each slot (2 * sizeof(int) hash check, sizeof(int) counter)
  \item Hash algorithm: DJBX33A (hash = ((hash<<5) + hash) + *key++), used by PHP, Apache and more. Use three seeds for position and check hash.
  \item Hash collision: Linearly find the next empty slot. This is simple (I'm afraid of pointers) and memory-saving.
\end{itemize}
\end{frame}

\begin{frame}{The End}
\begin{itemize}
  \item I intended to finish it in 3 days, but because I'm unfamiliar with these tools, it takes me a week (or more if including preparation time).
  \item Access logs are the origin of discoveries on access patterns. Data is precious, while disk is cheap. Please do not delete them after 52 days.
  \item I cannot draw conclusion on `trends', for there is not data for an adequately long time.
  \item If you want other statistics, please email me and I will query it (if I have time :).
\end{itemize}
\end{frame}

\begin{frame}{写在最后}
\begin{itemize}
  \item Thanks all maintainers and supporters of mirrors.ustc.edu.cn!
  \item All scripts and source of this slides are available at GitHub: https://github.com/bojieli/mirrors-log
  \item 终于搞定了中文字体问题,不过懒得翻译了……
\end{itemize}
\end{frame}

\end{document}
